\chapter[Nociones de Termodin'amica]{Nociones b'asicas de Termodin'amica relativista}\label{cap:termo}
\section{Primera ley de la Termodin'amica}
El primer objetivo de esta secci'on ser'a escribir la generalizaci'on relativista de la primera ley de la termodin'amica newtoniana (en equilibrio),
\begin{equation}\label{1termoa}
dU=-PdV+TdS,
\end{equation}
en el contexto de Relatividad General. Aqu'i, $dU$ es la \textit{energ'ia interna} total en el elemento de volumen $dV$ conteniendo un n'umero (constante) $\mathcal{N}$ bariones, $P$ es la presi'on, $T$ la temperatura y $S$ la entrop'ia total en dicho elemento de volumen. Para escribir la generalizaci'on apropiada de \eqref{1termoa} referimos las cantidades a un \textit{SR com'ovil} con el elemento de fluido $dV$. Con este objetivo, definimos la \textit{densidad num'erica propia de bariones} (los que contribuyen principalmente a la masa total de una estrella, por ejemplo) definida en este sistema de referencia:
\begin{equation}\label{densidadbarionica}
n=\frac{d\mathcal{N}}{dV}.
\end{equation}
Si la \textit{densidad de energ'ia interna} total es $\epsilon$, 
%relacionada con la \textit{densidad de masa} total de la materia $\rho$ por
%\begin{equation}
%\rho\,c^2=\epsilon,
%\end{equation}
entonces es conveniente escribir todas las cantidades que aparecen en la ley de conservaci'on de energ'ia \eqref{1termoa} en funci'on de $n$, ya que es una cantidad conservada. Es decir, en  t'erminos de la \textit{densidad de energ'ia interna por bari'on} $\epsilon/n$, el inverso de densidad de bariones $n^{-1}$ (volumen por bari'on), y la entrop'ia por bari'on $s=S/\mathcal{N}$, tendremos que \eqref{1termoa} se generaliza a:

\begin{align}\label{1ley_termo_b}
\boxed{d\left(\frac{\epsilon}{n}\right)=-Pd\left(\frac{1}{n}\right)+Tds.}
\end{align}
% \begin{equation}
% u=\frac{dU}{dV}\quad\Rightarrow\quad U=\frac{u A}{n},
% \end{equation}
% Por lo tanto, si la entrop'ia por bari'on es $s$ la ecuaci'on \ref{1termoa} se puede expresar como:
Si separamos expl'icitamente los diferenciales y agrupamos, notamos que
\begin{align}
\frac{d\epsilon}{n}+\epsilon d\left(\frac{1}{n}\right)+Pd\left(\frac{1}{n}\right)&=Tds\\
\frac{d\epsilon}{n}+(\epsilon +P)\left(-\frac{1}{n^2}\,dn\right)=Tds.
\end{align}
As'i, encontramos la siguiente relaci'on proveniente de la primera ley de la Termodin'amica relativista:
\begin{equation}\label{1leytermo}
\boxed{d\epsilon=\frac{\epsilon+P}{n}dn+nT\,ds.}
\end{equation}
La ecuaci'on \eqref{1ley_termo_b} tambi'en se puede reescribir en t'erminos de la densidad de energ'ia interna $u$. Para ello, primero definimos la \textit{densidad de masa en reposo} $\rho$ en t'erminos de la densidad num'erica de bariones:
\begin{equation}\label{masa_reposo}
\rho_0=m_B\,n
\end{equation}
en donde $m_B$ es la masa en reposo caracter'istica de un bari'on.
%, que para efectos pr'acticos, se puede considerar igual a la unidad de masa at'omica $m_u=1.66\cdot 10^{-27}\,[kg]$. 
De este modo, dado que la densidad de masa total $\rho$ incluye tanto la energ'ia interna como la energ'ia en reposo, tendremos:
\begin{equation}\label{energia_interna_y_densidad}
\boxed{u=\rho_0 c^2+\epsilon \quad\Leftrightarrow\quad \epsilon=u-\rho_0c^2=u-m_B nc^2}
\end{equation}
Entonces, dado que
\begin{equation}
d\left(\frac{\epsilon}{n}\right)=d\left(\frac{u-m_B\,n\,c^2}{n}\right)=d\left(\frac{u}{n}\right)-\cancelto{0}{d(m_B\,c^2)},
\end{equation}
tendremos que la ley \eqref{1ley_termo_b} equivale a:
\begin{align}
% d\left(\frac{u \cancel{A}}{n}\right)&=-Pd\left(\frac{\cancel{A}}{n}\right)+Td(\cancel{A}s)\\
\boxed{d\left(\frac{u}{n}\right)=-Pd\left(\frac{1}{n}\right)+Tds.}\label{1ley_termo}
\end{align}
De lo anterior es posible encontrar el an'alogo de \eqref{1leytermo}, pero usando $u$ en lugar de $\rho$:
\begin{align}
  du=\frac{u+P}{n}dn + nT\,ds.
\end{align}
Tambi'en podemos notar de las dos formas de la primera ley en \eqref{1ley_termo} y \eqref{1ley_termo_b}, que la presi'on y temperatura se pueden expresar como:
\begin{align}
 P&:=-\frac{\partial (\epsilon/n)}{\partial(1/n)}=-\frac{\partial (u/n)}{\partial(1/n)} \label{def_presion},\\
T&:=\frac{\partial (\epsilon/n)}{\partial s}=\frac{\partial (u/n)}{\partial s}\label{def_temperatura}.
\end{align}
% Esta ecuaci'on es conveniente interpretarla del siguiente modo, en vista de la discusi'on sobre ecuaciones de estado desarrolladas en la secci'on (FALTA)
% \begin{equation}
% \rho(n,s)=\rho[P(n,s),T(n,s)].
% \end{equation}
% Es decir, se puede deducir $P$ y $T$ a partir de $\rho$

\section{Ecuaci'on de estado adiab'atica o politr'opica}
Una ecuaci'on de estado adiab'atica o politr'opica es aquella en que la presi'on $P$ y la densidad num'erica de bariones $n$ ('o la densidad de masa en reposo $\rho_0$) se relacionan a trav'es de
\begin{equation}\label{estadopolitropica_general}
 P=K'n^{\gamma}\qquad\Leftrightarrow\qquad P=K\rho_0^{\gamma},
\end{equation}
en donde $K$ y $\gamma$ son constantes, siendo esta 'ultima el \textit{'indice adiab'atico 'o politr'opico}. Si $f$ es el n'umero de grados de libertad de las part'iculas constituyentes del fluido del que se compone la estrella, entonces $\gamma=(f+2)/f$. 

Es posible mostrar que esta relaci'on equivale a una ecuaci'on de estado que relacione la energ'ia interna $u$ con la presi'on $P$, s'olo suponiendo que sean proporcionales en la forma
\begin{equation}\label{estadopolitropica_alterna}
\boxed{ u=\frac{P}{\gamma-1},}
\end{equation}
y adem'as, que la entrop'ia por bari'on $s$ sea constante. En efecto, consideremos la primera ley de la termodin'amica en la forma \eqref{1ley_termo} y reemplacemos la relaci'on anterior:
\begin{align}
 \frac{1}{\gamma-1}\left[d\left(\frac{P}{n}\right)\right]+P\,d\left(\frac{1}{n}\right)&=0,\\
\frac{1}{\gamma-1}\left[\left(\frac{1}{n}\right)dP+P\,d\left(\frac{1}{n}\right)\right]+P\,d\left(\frac{1}{n}\right)&=0,\\
\frac{1}{\gamma-1}\left[\left(\frac{1}{n}\right)dP+\gamma P\, d\left(\frac{1}{n}\right)\right]&=0.
\end{align}
Multiplicando por $1/n^{\gamma-1}$, podemos escribir
\begin{align}
\frac{1}{n^{\gamma}}dP+\frac{\gamma P}{n^{\gamma-1}}\,d\left(\frac{1}{n}\right)&=0,\\
d\left(\frac{P}{n^{\gamma}}\right)&=0,
\end{align}
que, tal como se quer'ia probar, equivale a \eqref{estadopolitropica_general}. En el caso newtoniano, se tiene que la energ'ia interna es mucho menor que la densidad de masa en reposo $\epsilon\ll \rho_0 c^2$, por lo que de \eqref{energia_interna_y_densidad} tendremos
\begin{equation}
u\approx \rho_0c^2=m_Bn\,c^2,
\end{equation}
de donde recuperamos la forma usual de la ecuaci'on de estado politr'opica
\begin{equation}\label{estadopolitropica}
 \boxed{P=K\rho^{\gamma}.}
\end{equation}

\section{Involucrando la Temperatura y Entrop'ia}

En principio, el requisito m'inimo necesario para una ecuaci'on de estado del tipo $P=P(\rho)$ (usada para resolver el sistema de ecuaciones de estructura estelar), se basa en conocer c'omo se relacionan la presi'on y la densidad a partir de un par'ametro, que por simplicidad se escoge como la densidad num'erica de bariones $n$, pues en dicho caso recuperamos (\ref{estado}):
\begin{equation*}
 P=P(n),\qquad\rho=\rho(n)\quad\Rightarrow\quad n=n(\rho)\qquad\Rightarrow\quad P=P(n(\rho))=P(\rho).
\end{equation*}
Sin embargo, no se puede deducir en general $P$ ni $\rho$ a partir 'unicamente de un conocimiento de $n$, sino que se requiere adem'as la temperatura, $T$, o la entrop'ia por bari'on, $s$. De este modo, de las leyes de la termodin'amica es posible determinar todas las cantidades termodin'amicas restantes, a trav'es de las ecuaciones de estado m'as generales:
\begin{align}\label{estadogeneral}
 &P=P(n,s),\qquad\rho=\rho(n,s)\qquad\text{'o}\\
&P=P(n,T),\qquad\rho=\rho(n,T).
\end{align}

Sin embargo, para poder pasar de las ecuaciones anteriores, que involucran cantidades desconocidas como la temperatura  y la entrop'ia, a la ecuaci'on est'andar (\ref{estado}), se necesita informaci'on sobre las propiedades t'ermicas de la estrella. Por ejemplo, se puede considerar los efectos t'ermicos en la ecuaci'on de estado, al no  despreciar la temperatura de una estrella, de modo que:
\begin{equation}\marginnote{Efectos de la temperatura}
 P(r)=P(\rho(r),T(r)).
\end{equation}
Pero ahora necesitamos otra ecuaci'on adicional que ligue el campo escalar temperatura en funci'on del resto de las variables. Usualmente, se escoge para este prop'osito la ecuaci'on de energ'ia $E$, de modo de obtener la relaci'on deseada entre $P$ y $\rho$,
\begin{align}
 E(r)&=E(\rho(r),T(r)),
\end{align}
con $E$ conocida. Entonces, podemos escribir:
\begin{align}
T(r)&=T(E(r),\rho(r))=T(\rho(r)),\\
P(r)&=P(\rho(r),T(\rho(r)))=P(\rho(r)).
\end{align}

Pero esta 'ultima ecuaci'on de estado tambi'en ser'a 'util para modelar estrellas que est'an en la aproximaci'on \marginnote{Tipos de estrellas donde es v'alido $P=P(\rho)$}del cero absoluto, en las cuales la temperatura no juega ning'un rol relevante, tales como enanas blancas o estrellas de neutrones. Para este caso, de acuerdo al \textit{teorema de Nerns't}, la entrop'ia por nucle'on $s$ ser'a constante por toda la estrella, lo que es consistente con la reducci'on de las ecuaciones de estado generales (\ref{estadogeneral}) a (\ref{estado}):
\begin{equation}
 P=P(n,s=cte)=P(n),\qquad\rho=\rho(n,s=cte)=\rho(n)\qquad\Rightarrow P=P(\rho).
\end{equation}
Puede probarse\footnote{Ver Weinberg \cite{Weinberg72}.} tambi'en que la condici'on de entrop'ia constante, que seg'un la relaci'on anterior siempre conducir'a a una ecuaci'on de estado que relacione la presi'on y la densidad, se da tambi'en en estrellas en \textit{equilibrio convectivo}, en las cuales el mecanismo m'as eficiente para transferencia de energ'ia al interior de la estrella es convecci'on. Generalmente, esta condici'on se produce en estrellas supermasivas.

\section[Ecuaciones de estado de Fermi]{Ecuaciones de estado para un gas de Fermi completamente degenerado}\label{sec:ecsdeestado}
\subsection{Definiciones estad'isticas}
En esta secci'on se usar'an resultados de la teor'ia cin'etica para encontrar ecuaciones de estado apropiadas para modelar dos tipos de estrellas en las etapas finales de su evoluci'on: \textit{enanas blancas} y \textit{estrellas de neutrones}.

Consideremos en primer lugar la \textit{densidad num'erica en el espacio fase} 6-D para cada especie de part'icula:
\begin{equation}
 \frac{d\mathcal{N}}{d^3x\,d^3p}=\frac{g}{h^3}f(\vec{x},\vec{p},t)
\end{equation}
en donde $f(\vec{x},\vec{p},t)$ es la \textit{funci'on de distribuci'on adimensional en el espacio fase} que da el n'umero de ocupaci'on promedio en una celda de dicho espacio, $h$ es la constante de Planck de modo que $h^3$ sea el volumen de una celda en el espacio fase, y $g$ es el peso estad'istico  (n'umero de estados de una part'icula con un valor dado de momentum $\vec{p}$). Para part'iculas masivas, $g=2S+1$ ($S$ es el spin), para fotones $g=2$ y para neutrinos $g=1$.

Con esta definici'on, la densidad num'erica $n$ de cada especie de part'iculas ser'a:
\begin{equation}\label{densidad_dist}
 n=\int\frac{d\mathcal{N}}{d^3x\,d^3p}d^3p=\frac{g}{h^3}\int f\,d^3p,
\end{equation}
en donde la integral es sobre todo el espacio de los momenta $\vec{p}$. Por otra parte, la densidad de energ'ia $\epsilon$ estar'a dada por
\begin{equation}\label{energia_dist}
 u=\int E\frac{d\mathcal{N}}{d^3x\,d^3p}d^3p=\frac{g}{h^3}\int \left(p^2c^2+m^2c^4\right)^{1/2}f\,d^3p,
\end{equation}
en donde $E=\sqrt{p^2 c^2+m^2 c^4}$ es la energ'ia relativista de las part'iculas con masa en reposo $m$. Por otra parte, la presi'on $P$ para una \textit{distribuci'on isotr'opica de momenta} ser'a:
\begin{align}
 P&=\frac{1}{3}\int pv\left(\frac{d\mathcal{N}}{d^3x\,d^3p}\right)d^3p=\frac{g}{3h^3}\int \frac{p^2c^2}{E}f\,d^3p
 =\frac{g}{3h^3}\int p\frac{dE}{dp}f\,d^3p\label{presion_dist2}
\end{align}
en donde $v=pc^2/E$ y el factor $\frac{1}{3}$ proviene de la isotrop'ia considerada y el principio de equipartici'on. Finalmente, la densidad de masa en reposo $\rho$ se puede definir de dos formas para los casos de inter'es tratados:
\begin{enumerate}
 \item \emph{Electrones (en enanas blancas)}:
\begin{equation}\label{densidad_electrones}
\rho_0=\mu_e  m_u n_e,
\end{equation}
en donde $n=n_e$ es la \textit{densidad num'erica de electrones}, $m_u=1.66\cdot10^{-27}\,[kg]$ es la unidad de masa at'omica (e.d., de protones y de neutrones, ver ap'endice \ref{app:constantes}), y $\mu_e$ es el \textit{peso molecular medio por electr'on}. La expresi'on (\ref{densidad_electrones}) considera que la masa y por tanto la densidad del fluido considerado se debe a los bariones de los n'ucleos at'omicos, despreci'andose la contribuci'on de los electrones:
\begin{align}
 \mu_e&=\frac{\text{masa total}}{m_u}\frac{1}{N^{\circ}\text{ total de electrones}},\\
%&=\frac{m_B}{m_u Y_e}=\left(\frac{\sum_i n_i m_i}{\sum_i n_i A_i}\right)\frac{1}{m_u}\frac{1}{Y_e},\\
&\approx \frac{1}{Y_e}=\frac{N^{\circ}\text{ total de bariones}}{N^{\circ}\text{ total de electrones}}=\frac{Z}{A},
\end{align}
en donde $m_B\approx m_u$ es la \textit{masa media de bariones}, $Y_e$ es el \textit{n'umero medio de electrones por bari'on}, $A$ es el peso at'omico y $Z$ el n'umero at'omico de la especie considerada (e igual al n'umero de protones en el n'ucleo).  Para muchos de los elementos de los que usualmente est'a compuesta una enana blanca,  en donde esta relaci'on es v'alida, tal como ${}^4\textrm{He}$, ${}^{12}\textrm{C}$ y ${}^{24}\textrm{Mg}$, el peso molecular medio por electr'on es $\mu_e=2$. Una excepci'on es ${}^{56}\textrm{Fe}$, que tiene $\mu_e\approx2.15$.

\item \emph{Neutrones (en estrellas de neutrones)}:
\begin{equation}\label{densidad_neutrones}
\rho_0=  m_n n_n,
\end{equation}
en donde $n=n_n$ es la densidad num'erica de neutrones y $m_n$ es la masa de un neutr'on. Esto se debe a que en las estrellas de neutrones existen pr'acticamente s'olo estos bariones, por lo que la masa total ser'a debido 'unicamente a ellos.
\end{enumerate}

\subsection{Funci'on de distribuci'on de Fermi}
En general, de la estad'istica de Fermi-Dirac, sabemos que la funci'on de distribuci'on de un gas ideal de Fermi (describiendo fermiones, part'iculas de spin semi-entero) estar'a dada, como funci'on de la energ'ia, por:
\begin{equation}
 f(E)=\frac{1}{e^{\frac{E-\mu}{kT}}+1},
\end{equation}
en donde $\mu$ es el \textit{potencial qu'imico}. Es posible probar que la relaci'on anterior se reduce a la conocida funci'on de distribuci'on de Maxwell-Boltzmann para densidades bajas y temperaturas altas. Por otra parte, cuando las temperaturas son bajas, como ocurre con la materia presente en las estrellas analizadas, los fermiones se ir'an a los niveles de energ'ia m'as bajos disponibles, denomin'andose \textit{gas de Fermi completamente degenerado} en el l'imite $T\to 0$. Para este caso, el potencial qu'imico $\mu=E_f$ pasa a denominarse energ'ia de Fermi, y la funci'on de distribuci'on se convierte en una funci'on escal'on:
\begin{equation}\label{fermidegenerada}
f(E)=\begin{cases}
   1& \text{si } E \leq E_F ,\\
   0& \text{si } E > E_F.
  \end{cases}
\end{equation}
En esta situaci'on, todos los fermiones al estar ocupando el nivel m'as bajo, tendr'an $\left|\vec{p}\right|\le p_F$, en donde $p_F$ se denomina momentum de Fermi, estando relacionados con la energ'ia de Fermi mediante la relaci'on
\begin{equation}\label{energiafermi}
E_F=\sqrt{p_F^2 c^2+m_e^2 c^4}.
\end{equation}

\subsection{Ecuaci'on de estado de Fermi exacta}

\subsubsection{Densidad num'erica y densidad propia de masa}
Usando la funci'on de distribuci'on para un gas ideal de Fermi completamente degenerado \eqref{fermidegenerada} de electrones (con sub'indice $e$) o neutrones (con sub'indice $n$)  en la definici'on \eqref{densidad_dist}, obtenemos para la densidad num'erica de las part'iculas.
\begin{equation}
 n_{e,n}=\frac{2}{h^3}\int_0^{p_F}4\pi p^2 dp=\frac{8\pi}{3h^3}\,p_F^3,
\end{equation}
en donde se ha considerado $g=2$ debido a que el spin de los fermiones considerados, electrones y neutrones, es $S=1/2$. Definiendo el \textit{par'ametro adimensional de momentum relativo} $x$:
\begin{equation}\label{xrelativo}
 x:=\frac{p}{m_{e,n}c}\qquad\Rightarrow\qquad x_F=\frac{p_F}{m_{e,n} c},
\end{equation}
podemos escribir la relaci'on anterior como:
\begin{equation}\label{densidadfermi1}
 n_{e,n}=\frac{8\pi c^3}{3 h^3}m_{e,n}^3x_F^3.
\end{equation}
De este modo, podemos expresar la densidad propia de masa para el caso de electrones usando \eqref{densidad_electrones} como:
\begin{equation}\label{densidad_electrones-fermi}
\boxed{ \rho_0= \frac{8\pi c^3}{3 h^3}\mu_em_u m_e^3x_F^3\approx9.7393\cdot10^8\,\mu_e\, x_F^3\;[kg/m^3],}
\end{equation}
mientras que para neutrones usamos \eqref{densidad_neutrones}, obteniendo:
\begin{equation}\label{densidad_neutrones-fermi}
\boxed{ \rho_0= \frac{8\pi c^3}{3 h^3} m_n^4x_F^3\approx6.1066\cdot10^{18}\,x_F^3\;[kg/m^3].}
\end{equation}

\subsubsection{Calculando la presi'on y densidad de energ'ia}
Por otra parte, para la presi'on usamos la definici'on \eqref{presion_dist2}, obteniendo la integral
\begin{equation}
 P_{e,n}=\frac{1}{3}\frac{2}{h^3}\int_0^{p_F}\frac{p^2c^2}{(p^2 c^2+m_{e,n}^2 c^4)^{1/2}}4\pi p^2\,dp=\frac{8\pi c^2}{3h^3}\int\limits_0^{p_F}\frac{p^4\,dp}{(p^2 c^2+m_e^2c^4)^{1/2}},
\end{equation}
y en funci'on del par'ametro relativo $x$ \eqref{xrelativo}, podemos escribir la integral de presi'on en la forma:
\begin{equation}\label{presionfermi1}
P_{e,n}=\frac{8\pi m_{e,n}^4c^5}{3h^3}\int\limits_0^{x_F}\frac{x^4\,dx}{(1+x^2)^{1/2}}.
\end{equation}
Adem'as, la segunda forma en que se ha escrito la integral de presi'on \eqref{presion_dist2}  permite encontrar una relaci'on directa con la integral de energ'ia \eqref{energia_dist}, puesto que integrando por partes con la funci'on de distribuci'on considerada \eqref{fermidegenerada}, podemos escribir:
\begin{align}
 P=\frac{2}{3h^3}\int\limits_0^{p_F} p\frac{dE}{dp}\,d^3p&=\frac{8\pi}{3h^3}\int\limits_0^{p_F} p^3\frac{dE}{dp}\,dp\\
&=\frac{8\pi}{3h^3}\left\{\int\limits_0^{p_F} \frac{d}{dp}\left(p^3 E\right)\,dp-\int\limits_0^{p_F} 3p^2E\,dp\right\}\\
&=\left.\frac{8\pi}{3h^3}p^3E\right|_0^{pf}-\frac{8\pi}{h^3}\int\limits_0^{p_F}Ep^2\,dp\\
&=\frac{8\pi}{3h^3}p_F^3E_F-u\label{integral-presion-energia-fermi}.
\end{align}
En t'erminos del par'ametro relativo $x_F$, tenemos para la densidad de energ'ia, que:
\begin{align}
u_{e,n}&=\frac{8\pi}{3h^3}p_F^3E_F-P_{e,n}\\
&=\frac{8\pi}{3h^3}\left(\frac{p_F}{m_{e,n}c}\right)^3\left(m_{e,n}c\right)^3\sqrt{p_F^2c^2+m_{e,n}^2c^4}-P_{e,n}\\
&=\frac{8\pi m_{e,n}^4c^5}{3h^3}x_F^3\sqrt{1+x_F^2}-P_{e,n}.\label{fermi-relacion-presion-energia}
\end{align}
Por lo tanto, s'olo basta determinar la integral de presi'on \eqref{presionfermi1} para obtener de la expresi'on anterior la densidad de energ'ia $\epsilon$, no requiriendo calcular directamente \eqref{energia_dist}.
% Aunque por completitud, igual se dar'a la forma expl'icita de la integral de energ'ia:
% \begin{align}
%  \epsilon_{e,n}&=\frac{2}{h^3}\int\limits_0^{p_F}\left(p^2 c^2+m_{e,n}^2 c^4\right)4\pi p^2\,dp\\
% &=\frac{8\pi m_{e,n}^4c^5}{h^3}\int\limits_0^{x_F}x^2\sqrt{x^2+1}\,dx.\label{energiafermi1}
% \end{align}
Entonces, para calcular \eqref{presionfermi1}, conviene efectuar la sustituci'on hiperb'olica,
\begin{align}\label{sust-hiperbolica}
 x&=:\senh\theta\quad\Rightarrow\quad dx=\cosh\theta d\theta,\qquad \left(1+x^2\right)^{1/2}=\cosh\theta,\\
\theta_F&=\senh^{-1}x_F=\ln\left|x_F+\sqrt{1+x_F^2}\right|,
\end{align}
con la cual la integral de presi'on queda
\begin{align}\label{presionfermi-hiperbolica1}
 P_{e,n}=\frac{8\pi m_{e,n}^4c^5}{3h^3}\int\limits_0^{\theta_F}\senh^4\theta\, d\theta.
\end{align}
Para determinarla, usamos algunas identidades hiperb'olicas\footnote{$\cosh^2\theta-\senh^2\theta\equiv 1$, $\senh2\theta\equiv 2\senh\theta\cosh\theta$, $\cosh2\theta\equiv \senh^2\theta+\cosh^2\theta\equiv 1+2\senh^2\theta$, $\senh^2\theta\equiv \left(\cosh2\theta-1\right)/2$.} de modo que el integrando se pueda reescribir como:
\begin{align}
 \senh^4\theta&=\senh^2\theta\left(\cosh^2\theta-1\right)=\left(\frac{\senh^2 2\theta}{2}\right)^2-\senh^2\theta,\\
&=\frac{1}{4}\left(\frac{\cosh4\theta-1}{2}\right)-\left(\frac{\cosh2\theta-1}{2}\right),\\
&=\frac{1}{8}\left(\cosh4\theta-4\cosh2\theta+3\right)\label{fermi-senh4},
\end{align}
con lo cual se puede calcular directamente la integral \eqref{presionfermi-hiperbolica1}, ya que al reemplazar all'i \eqref{fermi-senh4}, obtenemos:
\begin{align}
 P_{e,n}&=\frac{\pi m_{e,n}^4c^5}{3h^3}\int\limits_0^{\theta_F}\left(\cosh4\theta-4\cosh2\theta+3\right)\, d\theta\\
&=\frac{\pi m_{e,n}^4c^5}{3h^3}\left(\frac{1}{4}\senh4\theta_F-2\senh2\theta_F+3\theta_F\right) .\label{presionfermi-hiperbolica2}
\end{align}
Esta expresi'on se puede escribir de varias formas equivalentes, las que ser'an 'utiles seg'un las circunstancias:
\begin{itemize}
 \item Factorizando directamente \eqref{presionfermi-hiperbolica2} por 1/4,
\begin{align}
 P_{e,n}=\frac{1}{3}\frac{\pi m_{e,n}^4c^5}{4h^3}\left[\senh\left(4\theta_F\right)-8\senh\left(\frac{4\theta_F}{2}\right)+3\left(4\theta_F\right)\right],
\end{align}
e introduciendo el par'ametro $t$ definido por
\begin{align}\label{fermi-parametro-t}
 t:=4\theta_F&=4\senh^{-1}x_F=4\ln\left|x_F+\sqrt{1+x_F^2}\right|,
\end{align}
obtenemos la forma param'etrica para la presi'on dada por Oppenheimer \cite{Oppenheimer39enero}:
\begin{align}\label{fermi-presion-OV}
 \boxed{P_{e,n}=\frac{1}{3}\frac{\pi m_{e,n}^4c^5}{4h^3}\left(\senh t-8\senh\left(\frac{t}{2}\right)+3t\right).}
\end{align}
Para encontrar la densidad de energ'ia, usamos la relaci'on \eqref{fermi-relacion-presion-energia}, notando que de la sustituci'on \eqref{sust-hiperbolica} podemos escribir
\begin{align}
 x_F^3\sqrt{1+x_F^2}&=\senh^3\theta_F\cosh\theta_F=\left(\senh^2\theta_F\right)\left(\senh\theta_F \cosh\theta_F\right)\\
&=\left(\frac{\cosh 2\theta_F-1}{2}\right)\left(\frac{\senh 2\theta_F}{2}\right)
% \frac{1}{4}\left(\cosh 2\theta_F\senh 2\theta_F\right)-\frac{1}{4\senh 2\theta_F}\\
=\frac{1}{8}\senh4\theta_F-\frac{1}{4}\senh2\theta_F.
\end{align}
Luego, obtenemos que:
\begin{align}
u_{e,n}&=\frac{8\pi m_{e,n}^4c^5}{3h^3}\left[\frac{1}{8}\senh4\theta_F-\frac{1}{4}\cancel{\senh2\theta_F}\right]-\frac{1}{3}\frac{\pi m_{e,n}^4c^5}{4h^3}\left[\senh\left(4\theta_F\right)-8\cancel{\senh\left(\frac{4\theta_F}{2}\right)}+3\left(4\theta_F\right)\right],\\
&=\frac{\pi m_{e,n}^4c^5}{3h^3}\left[\frac{3}{4}\senh 4\theta_F-3\theta_F\right],
\end{align}
y en t'erminos del par'ametro $t$ \eqref{fermi-parametro-t}:
\begin{align}\label{fermi-energia-OV}
 \boxed{u_{e,n}=\frac{\pi m_{e,n}^4c^5}{4h^3}(\senh t-t).}
\end{align}

\item Tambi'en podemos expresar \eqref{presionfermi-hiperbolica2} directamente en t'erminos del par'ametro $x_F$, para lo cual se reescribe dicha expresi'on en la forma:
\begin{align}
P_{e,n}&=\frac{\pi m_{e,n}^4c^5}{3h^3}\left(\frac{1}{2}\senh2\theta_F\cosh2\theta-2\senh2\theta_F+3\theta_F\right),\\
&=\frac{\pi m_{e,n}^4c^5}{3h^3}\left(\frac{1}{2}\senh2\theta_F\left(1+2\senh^2\theta_F\right)-2\senh2\theta_F+3\theta_F\right),\\
&=\frac{\pi m_{e,n}^4c^5}{3h^3}\left(\senh\theta_F\cosh\theta_F\left(2\senh^2\theta_F-3\right)+3\theta_F\right),
\end{align}
y reexpresando en t'erminos de $x_F$ mediante \eqref{sust-hiperbolica}, obtenemos para la presi'on
\begin{equation}\label{presionfermi2}
\boxed{ P_{e,n}=\frac{\pi m_{e,n}^4c^5}{3h^3}\left[x_F\sqrt{1+x_F^2}\left(2x_F^2-3\right)+3\ln\left|x_F+\sqrt{1+x_F^2}\right|\right],}
\end{equation}
\end{itemize}
y mediante \eqref{fermi-relacion-presion-energia}, obtenemos directamente para la densidad de energ'ia la siguiente relaci'on:
\begin{equation}\label{energiafermi2}
 \boxed{u_{e,n}=\frac{\pi m_{e,n}^4c^5}{h^3}\left[x_F\sqrt{1+x_F^2}\left(1+2x_F^2\right)-\ln\left|x_F+\sqrt{1+x_F^2}\right|\right],}
\end{equation}
que son las formas para estas variables dadas por Chandrasekhar \cite{Chandra39} y Shapiro \cite{Shapiro83}.


\subsubsection{Obtenci'on de la ecuaci'on de estado expl'icita}

De esta forma, las ecuaciones \eqref{densidad_electrones} 'o \eqref{densidad_neutrones}, \eqref{presionfermi2} y \eqref{energiafermi2} proporcionar'an una forma param'etrica para la ecuaci'on de estado de Fermi completamente degenerada (y exacta) en funci'on de $x_F$: $\rho_0=\rho_0(x_F)$, $P=P(x_F)$ y $u=u(x_F)$.

Ahora bien, para determinar una ecuaci'on de estado del tipo \eqref{estado}, se despeja $x_F$ de \eqref{densidad_electrones-fermi}, de modo de obtener una dependencia con la densidad propia de masa del tipo $x_F=x_F(\rho)$, que para el caso de electrones es:
\begin{equation}\label{xrelativo_electrones}
 x_F=\left(\frac{3h^3}{8\pi c^3 m_u m_e^3}\right)^{1/3}\left(\frac{\rho_0}{\mu_e}\right)^{1/3}\approx\left(\frac{\rho_0/\mu_e}{9.7393\cdot10^8\,[kg/m^3]}\right)^{1/3},
\end{equation}
y para neutrones ser'ia:
\begin{equation}\label{xrelativo_neutrones}
 x_F=\left(\frac{3h^3}{8\pi c^3 m_n^4}\right)^{1/3}\left(\rho_0\right)^{1/3}\approx\left(\frac{\rho_0/\mu_e}{6.1066\cdot10^{18}\,[kg/m^3]}\right)^{1/3}.
\end{equation}

Luego, una ecuaci'on de estado que relacione la presi'on con la densidad de masa se obtendr'a sustituyendo lo anterior en \eqref{presionfermi2}, obteniendo $P=P(x_F)=P(x_F(\rho_0))=P(\rho_0)$. Del mismo modo, se puede encontrar la densidad de energ'ia en funci'on de la densidad propia de masa, usando \eqref{energiafermi2}: $u=\epsilon(x_F)=u(x_F(\rho_0))=u(\rho_0)$.

\subsection{Ecuaciones de estado de Fermi aproximadas}
Las expresiones resultantes para las ecuaciones de estado de Fermi exactas son poco manejables anal'iticamente, por lo que su soluci'on recae en m'etodos num'ericos, tal como se hizo para las ecuaciones de estructura estelar en la secci'on \ref{sec:fermi-exacta}. Por esta raz'on, y con el objetivo de simplificar la ecuaci'on de estado obtenida, consideraremos dos casos extremos para el par'ametro $x_F$ dado por \eqref{xrelativo}:

\begin{enumerate}
 \item $x_F\ll1$. Es posible encontrar directamente una serie de potencias para la presi'on \eqref{presionfermi2} y densidad de energ'ia \eqref{energiafermi2} a paritr de dichas expresiones. Sin embargo, es m'as f'acil expandir primero el integrando de \eqref{presionfermi1}, ya que en este l'imite:
\begin{align}
P_{e,n}&\approx\frac{8\pi m_{e,n}^4c^5}{3h^3}\int\limits_0^{x_F}x^4\left[1-\frac{1}{2}x^2+\left(-\frac{1}{2}\right)\left(-\frac{3}{2}\right)\left(\frac{1}{2!}\right)x^4+\cdots\right]\,dx,\\
&=\frac{8\pi m_{e,n}^4c^5}{3h^3}\left[\frac{x_F^5}{5}-\frac{1}{2}\frac{x_F^7}{7}+\frac{3}{8}\frac{x_F^9}{9}+\cdots\right],
\end{align}
y tambi'en, usando \eqref{fermi-relacion-presion-energia} y expandiendo:
\begin{align}
 u_{e,n}&\approx\frac{8\pi m_{e,n}^4c^5}{3h^3}\left[x_F^3\left(1+\frac{1}{2}x_F^2+\left(\frac{1}{2}\right)\left(-\frac{1}{2}\right)\left(\frac{1}{2!}\right)x_F^4+\cdots\right)\right]-P_{e,n}\\
&=\frac{8\pi m_{e,n}^4c^5}{3h^3}\left\{\left[x_F^3+\frac{1}{2}x_F^5-\frac{1}{8}x_F^7+\cdots\right]-\left[\frac{x_F^5}{5}-\frac{1}{2}\frac{x_F^7}{7}+\frac{3}{8}\frac{x_F^9}{9}+\cdots\right]\right\}\\
&=\frac{8\pi m_{e,n}^4c^5}{3h^3}\left[x_F^3+\frac{3}{10}x_F^5-\frac{3}{56}x_F^7+\cdots \right]
\end{align}
De este modo\footnote{Note que comparando ambas expansiones con \eqref{presionfermi2} y \eqref{energiafermi2}, es posible encontrar las siguientes relaciones va'lidas para $x_F\ll1$:
\begin{align}
 x_F\sqrt{1+x_F^2}\left(2x_F^2-3\right)+3\ln\left|x_F+\sqrt{1+x_F^2}\right|&=\frac{8}{5}\left(x_F^5-\frac{5}{14}x_F^7+\frac{5}{24}x_F^9+\cdots\right),\\
 x_F\sqrt{1+x_F^2}\left(1+2x_F^2\right)-\ln\left|x_F+\sqrt{1+x_F^2}\right|&=\frac{8}{3}\left(x_F^3+\frac{3}{10}x_F^5-\frac{3}{56}x_F^7+\cdots \right).
\end{align}
}, dejando s'olo el primer t'ermino en la expresi'on para la presi'on:
\begin{equation}\label{presionfermi1-asintotico}
P_{e,n}=\frac{8\pi m_{e,n}^4c^5}{3h^3}\frac{x_F^5}{5}
\end{equation}
obtenemos una ecuaci'on de estado politr'opica \eqref{estadopolitropica} con 'indice $\gamma=5/3$, ya que $x_F$ es proporcional a $\rho_0^{1/3}$. Su expresi'on expl'icita depender'a del tipo de part'icula involucrada:
\begin{enumerate}
\item \emph{Electrones no relativistas} ($p_F\ll m_{e}c$). Equivale por  \eqref{xrelativo_electrones} a que las densidades t'ipicas de las enanas blancas sean bajas, del orden de $\rho_0\ll10^{9}\,[kg/m^3]$. Reemplazando dicho $x_F$ en la expansi'on asint'otica de la presi'on \eqref{presionfermi1-asintotico}, tenemos:
\begin{equation}\label{fermi_norelativista}
 \boxed{P_e=\frac{3^{2/3}\pi^{4/3}}{5}\frac{\hbar^2}{m_e(m_u\mu_e)^{5/3}}\rho_0^{5/3}\approx1.00359\cdot 10^{7}\,\left(\frac{\rho_0}{\mu_e}\right)^{5/3}\,MKS.}
\end{equation}

\item \emph{Neutrones no relativistas} ($p_F\ll m_{n}c$). Equivale por  \eqref{xrelativo_neutrones} a que las densidades t'ipicas de las estrellas de neutrones sean bajas, del orden de $\rho\ll6\cdot10^{18}\,[kg/m^3]$.  Reemplazando dicho $x_F$ en la expansi'on asint'otica de la presi'on \eqref{presionfermi1-asintotico}, tenemos:
\begin{equation}\label{fermi_norelativista2}
 \boxed{P_n=\frac{3^{2/3}\pi^{4/3}}{5}\frac{\hbar^2}{m_n^{8/3}}\rho_0^{5/3}\approx5.3803\cdot 10^{3}\,\left(\rho_0\right)^{5/3}\,MKS.}
\end{equation}

\end{enumerate}
 \item $x_F\gg1$. Al igual que para el caso anterior, en vez de expandir directamente la serie de potencias para la presi'on \eqref{presionfermi2} y densidad de energ'ia \eqref{energiafermi2}, es m'as conveniente desarrollar primero el integrando de \eqref{presionfermi1}, que en este l'imite es:
\begin{align}
P_{e,n}&\approx\frac{8\pi m_{e,n}^4c^5}{3h^3}\int\limits_0^{x_F}\frac{x^4\,dx}{x\left(1+\frac{1}{x^2}\right)^{1/2}}\\
&=\frac{8\pi m_{e,n}^4c^5}{3h^3}\int\limits_0^{x_F}x^3
\left[1-\frac{1}{2}\frac{1}{x^2}+\left(-\frac{1}{2}\right)\left(-\frac{3}{2}\right)\left(\frac{1}{2!}\right)\frac{1}{x^4}+\cdots\right]\,dx,\\
&=\frac{8\pi m_{e,n}^4c^5}{3h^3}\left[\frac{x_F^4}{4}-\frac{1}{2}\frac{x_F^2}{2}+\frac{3}{8}\ln(x_F)+\cdots\right],
\end{align}
y tambi'en, usando \eqref{fermi-relacion-presion-energia} y expandiendo:
\begin{align}
 u_{e,n}&\approx\frac{8\pi m_{e,n}^4c^5}{3h^3}\left[x_F^4\left(1+\frac{1}{x_F^2}\right)^{1/2}\right]-P_{e,n}\\
&=\frac{8\pi m_{e,n}^4c^5}{3h^3}\left[x_F^4\left(1+\frac{1}{2}\frac{1}{x_F^2}+\left(\frac{1}{2}\right)\left(-\frac{1}{2}\right)\left(\frac{1}{2!}\right)\frac{1}{x_F^4}+\cdots\right)\right.\\
&\left.\quad-\left(\frac{x_F^4}{4}-\frac{1}{2}\frac{x_F^2}{2}+\frac{3}{8}\ln(x_F)+\cdots\right)\right]\\
&=\frac{8\pi m_{e,n}^4c^5}{3h^3}\left[\frac{3}{4}x_F^4+\frac{3}{4}x_F^2-\frac{3}{8}\ln(x_F)+\cdots\right]\label{energia_expansion_fermi_relativista}
\end{align}
De este modo\footnote{Note que comparando ambas expansiones con \eqref{presionfermi2} y \eqref{energiafermi2}, es posible encontrar las siguientes relaciones v'alidas para $x_F\gg1$:
\begin{align}
 x_F\sqrt{1+x_F^2}\left(2x_F^2-3\right)+3\ln\left|x_F+\sqrt{1+x_F^2}\right|&=2\left(x_F^4-x_F^2+\frac{3}{2}\ln(x_F)+\cdots\right),\\
 x_F\sqrt{1+x_F^2}\left(1+2x_F^2\right)-\ln\left|x_F+\sqrt{1+x_F^2}\right|&=2\left(x_F^4+x_F^2-\frac{1}{2}\ln(x_F)+\cdots \right).
\end{align}
}, dejando s'olo el primer t'ermino en la expresi'on para la presi'on:
\begin{equation}\label{presionfermi2-asintotico}
P_{e,n}=\frac{8\pi m_{e,n}^4c^5}{3h^3}\frac{x_F^4}{4}
\end{equation}
obtenemos una ecuaci'on de estado politr'opica \eqref{estadopolitropica} con 'indice $\gamma=4/3$, ya que $x_F$ es proporcional a $\rho_0.^{1/3}$. Su expresi'on expl'icita depender'a del tipo de part'icula involucrada:

\begin{enumerate}
\item \emph{Electrones ultra-relativistas} ($p_F\gg m_{e}c$). Equivale por  \eqref{xrelativo_electrones} a que las densidades t'ipicas de las enanas blancas sean altas, del orden de $\rho_0\gg10^{9}\,[kg/m^3]$. Reemplazando dicho $x_F$ en la expansi'on asint'otica de la presi'on \eqref{presionfermi2-asintotico}, tenemos:
\begin{equation}\label{fermi_relativista}
 \boxed{P_e=\frac{3^{1/3}\pi^{2/3}}{4}\frac{\hbar c}{(m_u\mu_e)^{4/3}}\rho_0^{4/3}\approx1.2435\cdot 10^{10}\,\left(\frac{\rho_0}{\mu_e}\right)^{4/3}\,MKS.}
\end{equation}

\item \emph{Neutrones ultra-relativistas} ($p_F\gg m_{n}c$). Equivale por  \eqref{xrelativo_neutrones} a que las densidades t'ipicas de las estrellas de neutrones sean altas, del orden de $\rho_0\gg6\cdot10^{18}\,[kg/m^3]$.Reemplazando dicho $x_F$ en la expansi'on asint'otica de la presi'on \eqref{presionfermi2-asintotico}, tenemos:
\begin{equation}\label{fermi_relativista2}
 \boxed{P_e=\frac{3^{1/3}\pi^{2/3}}{4}\frac{\hbar c}{m_n^{4/3}}\rho_0^{4/3}\approx1.2293\cdot 10^{10}\,\left(\rho_0\right)^{4/3}\,MKS.}
\end{equation}

\end{enumerate}
\end{enumerate}

La diferencia en el exponente ('indice politr'opico $\gamma$) de las ecuaciones de estado obtenidas tiene importancia fundamental en la estabilidad de enanas blancas y estrellas de neutrones, tanto aplicando la teor'ia newtoniana de gravitaci'on como Relatividad General.

Notar que en las aplicaciones del texto principal en que se usen resultados de este ap'endice, se omitir'a el sub'indice $F$ en el par'ametro relativo $x_F$, a fin de no recargar la notaci'on.